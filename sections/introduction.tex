
% \subsection{Contributions}
We introduce a novel verification case study for an aerial delivery cyber-physical system: a planar quadrotor with an underslung payload connected via a tether. The benchmark preserves key non-linear couplings between vehicle motion, payload swing, and tether dynamics while remaining sufficiently structured to integrate with current neural network verification tools. Using this benchmark, we develop an expert PID controller and train a neural network controller by behavioural cloning, providing a learned control policy suitable for both empirical evaluation and formal analysis.

To improve robustness and encourage safe behaviour, we investigate robustness-based training, including adversarial training under bounded input perturbations, and property-driven training (PDT), in which controller-level specifications are translated into differentiable penalty terms and integrated into the learning objective. We analyse how these training strategies affect both empirical robustness and verifiability. To support interpretability and diagnostic insight, we apply explainability analysis using LIME \cite{ribeiro_why_2016} and Integrated Gradients~\cite{sundararajan2017axiomatic} to compare feature importance and stability between baseline and robustly trained controllers.

Finally, we evaluate the learned controllers using three complementary assurance techniques at different levels of abstraction: (i) controller-level formal property verification of the neural network in isolation using the \vehicle\ specification language \cite{vehicle} with the \marabou\ solver \cite{wu2024marabou}; (ii) system-level finite-horizon closed-loop reachability analysis using CORA \cite{cora}; and (iii) empirical robustness evaluation via adversarial perturbation testing. All code for simulation, training, and verification is released as open source\footnote{\url{https://github.com/calumarnott/f21zk.git}}.


\subsection{Drone Delivery}
The use of Uncrewed Aerial Vehicles (UAVs)--more commonly known as drones--for the delivery of packages has become increasingly widespread in both rural and urban environments. Drone delivery has been used to improve access to healthcare and ensure time-critical delivery of medical goods, such as vaccines \cite{haidari_economic_2016,enayati_vaccine_2023,ospina-fadul_cost-effectiveness_2025} and blood \cite{nisingizwe_effect_2022,ling_aerial_2019,glauser_blood-delivering_2018,amukele_can_2015}. In urban environments, drones are increasingly being developed for last-mile delivery of food and consumer goods \cite{brunner_urban_2019,mazur_commercial_2025}. Operation of drones, particularly in urban environments, requires adherence to strict safety and reliability requirements administered by local flight aviation authorities.

Several mechanisms have been utilised for delivering packages using drones, including parachute-based delivery \cite{glauser_blood-delivering_2018,ho_optimizing_2025,hughes_autonomous_2023} and servo-actuated mechanisms \cite{maheshwar_development_2024,iqbal_servo_2016}. Whilst these approaches offer a straightforward method for dropping the payload (‘package’), neither offers precise control over the payload descent dynamics. Increasingly, winch based mechanisms are explored in which the payload is suspended beneath the drone on a tether and lowered to the ground using a motorised winch \cite{poulsen_uncrewed_2024,crowe_design_2025,pillai_optimizing_2025}. This configuration has several practical advantages, including the ability to deliver a payload without requiring a human to approach the drone itself, thereby reducing the risk of collision with hazardous rotating components such as propellers. Nevertheless, winched delivery introduces safety concerns to address, including payload swing, slack or excessive tension in the tether, winch overload, and collisions with the surrounding environment. 

The resulting drone-payload system shows coupled multi-body dynamics along with non-linear effects. Ensuring safe operation therefore requires careful control design and rigorous analysis.

\subsection{Neural Network Control of Cyber-Physical Systems}
Neural Networks (NNs) have been utilised for control in cyber-physical systems \cite{musa_deep_2023}, including for UAV applications \cite{amer_deep_2021,yu_quadrotor_2023}. NNs are appealing due to their ability to learn complex non-linear control mappings, to cope with uncertainty, and to perform well when accurate modelling of system dynamics is difficult \cite{gu_uav_2020}. These properties are particularly relevant in aerial robotics, where system dynamics include coupled translational and rotational motion, non-linear actuator behaviour, and environmental disturbances such as wind. However, the adoption of NN controllers in safety-critical applications remains limited by the lack of formal guarantees on their behaviour. Verification of NNs as controllers is a burgeoning research field \cite{sun_formal_2019,seshia_formal_2018,genin_formal_2022}, and formal verification techniques are desirable for assuring safety in systems that operate in close proximity to people and the environment, such as delivery drones.

From a verification perspective, the drone delivery system with an underslung payload presented has several distinguishing challenges from existing cyber-physical system (CPS) benchmarks. Firstly, the NN controller is based on a regression network, which produces continuous-valued control outputs rather than discrete actions or class labels. Moreover, the closed-loop dynamics of the combined drone–payload system are non-linear and relatively high-dimensional, making reachability analysis computationally challenging. Additionally, safety is not naturally expressed as simple avoidance of unsafe regions in the state space; rather, desired behaviour is more naturally characterised as maintaining bounded deviation from a target trajectory or operating regime. Though prior benchmarks have addressed the verification of neural network controllers for a quadcopter drone \cite{lopez_arch-comp23_2023}, the inclusion of an underslung payload increases the complexity of the system as the controller must simultaneously regulate the motion of the drone and the payload, which are highly coupled dynamically. 

This places the drone-payload system at the boundary of what existing neural network verification and reachability tools can handle in practice.

% \subsection{Contributions}

% This paper presents a case study on neural network verification for an aerial delivery system, consisting of a quadrotor UAV (drone) with a tethered payload. Focus is placed on evaluating existing verification tools and methodologies when applied to a novel CPS. To this end, we develop a NN controller for the drone-payload system using behaviour cloning from an expert Proportional-Integral-Derivative (PID) controller. We derive a planar model of the combined drone–payload dynamics, making explicit assumptions and simplifications that preserve the key non-linear couplings while simplifying the system sufficiently for analysis with the verification tools assessed.

% Robustness of the NN controller is investigated by analysing its sensitivity to bounded input perturbations and by applying adversarial training techniques. We apply explainable AI techniques like LIME \cite{ribeiro_why_2016} to analyse feature importance in the controller’s decision-making process and to compare baseline and adversarially trained networks. Based on our results, we identify where verification succeeds, where it fails, and what these outcomes imply for the verification of NN controllers in safety-critical aerial robotics. The insights gained from this case study highlight gaps between realistic CPS requirements and current verification tooling, and point towards directions for future benchmark design and tool development.

% We formalise a set of physically meaningful, controller-level safety properties, such as bounded actuation and correct control direction, which are interpretable from an engineering perspective. These properties are evaluated at the neural network level using Vehicle with the Marabou solver. In addition, we perform finite-time closed-loop reachability analysis of the NN-controlled system using CORA, highlighting the effects of nonlinear dynamics, dimensionality, and set explosion on verification scalability.

% All code developed for this case study, including the drone-payload simulation environment, neural network training scripts, and verification analyses, is made publicly available in our GitHub repository\footnote{\url{https://github.com/calumarnott/f21zk.git}}.




% This paper presents a case study on neural network verification for an aerial delivery system, consisting of a quadrotor UAV (drone) with a tethered payload. Focus is placed on evaluating existing verification tools and methodologies when applied to a novel CPS. To this end, we develop a NN controller for the drone-payload system using behaviour cloning from an expert Proportional-Integral-Derivative (PID) controller. We derive a planar model of the combined drone–payload dynamics, making explicit assumptions and simplifications that preserve the key non-linear couplings while simplifying the system sufficiently for analysis with the verification tools assessed.

% Robustness of the NN controller is investigated by analysing its sensitivity to bounded input perturbations and by applying adversarial training techniques. We apply explainable AI techniques like LIME \cite{ribeiro_why_2016} to analyse feature importance in the controller’s decision-making process and to compare baseline and adversarially trained networks. We also formalise a set of physically meaningful, controller-level safety properties, such as bounded actuation and correct control direction, which are interpretable from an engineering perspective. These properties are evaluated at the neural network level using Vehicle with the Marabou solver. In addition, we perform finite-time closed-loop reachability analysis of the NN-controlled system using CORA, highlighting the effects of nonlinear dynamics, dimensionality, and set explosion on verification scalability.

% All code developed for this case study, including the drone-payload simulation environment, neural network training scripts, and verification analyses, is made publicly available in our GitHub repository\footnote{\url{https://github.com/calumarnott/f21zk.git}}.




% Firstly, we verify properties of the neural network controller in isolation, independent of the system dynamics. This class of analysis focuses on infinite time-horizon properties of the controller itself, such as bounded control outputs and robustness to bounded input perturbations. We perform controller-level verification using the Marabou SMT-based neural network verifier, accessed through the Vehicle specification language. Vehicle provides a higher-level interface for expressing verification properties in physically meaningful units, thereby helping bridge the embedding gap between the physical domain and the vectorised representations required by neural network verifiers.

% Secondly, we analyse the closed-loop behaviour of the neural network controller together with the non-linear system dynamics using finite time-horizon reachability analysis. .This class of verification determines whether all trajectories originating from a specified initial set avoid unsafe regions over a bounded time interval. For this we use CORA , a framework implemented in MATLAB, which supports non-linear continuous-time dynamics and neural network controllers. This enables system-level safety analysis that cannot be addressed by controller-level verification alone.

% Finally, we incorporate robustness-oriented training techniques from the machine learning literature to improve both empirical performance and verifiability. In particular, we apply projected gradient descent (PGD)–based adversarial training to the neural network controller and compare a baseline model against an adversarially trained variant. The two controllers are evaluated with respect to performance under adversarial perturbations as well as their ability to satisfy formal verification properties. In addition, we explore property-driven training approaches that directly optimise the neural network to satisfy standard robustness specifications, enabling a direct comparison of verification outcomes before and after robustness-oriented training.
