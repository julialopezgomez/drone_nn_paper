This case study examined the robustness and verifiability of a neural network controller for a quadrotor with an underslung payload, highlighting both the potential and the current limitations of neural network verification tools when applied to realistic aerial robotics systems.

Our results show that controller-level formal verification using \vehicle\ and \marabou\ can successfully capture meaningful, physically interpretable properties such as bounded actuation, correct control direction, and quiet behaviour near equilibrium. Incorporating these properties into training via property-driven training measurably improved verifiability, with additional properties being formally satisfied on a benchmark problem. This demonstrates that aligning learning objectives with formal specifications can positively influence verification outcomes. However, scalability remains a key challenge: full controller-level verification of the tethered drone neural network was not tractable due to input dimensionality, dataset size, and solver constraints.

System-level reachability analysis using CORA further illustrated the difficulty of verifying closed-loop neural network–controlled systems with non-linear, coupled dynamics. While CORA provides valuable insight into finite-horizon behaviour, set explosion and computational limits restrict its applicability to higher-dimensional problems. Complementary adversarial robustness analysis supports this but does not offer formal guarantees. Explanation stability also improves, with LIME drift approximately halved and smaller Integrated Gradients under attack, indicating more consistent reliance on key features.

Our work highlights the gap between realistic cyber-physical control problems and the capabilities of current verification tooling. By combining behavioural cloning, robustness-oriented training, controller-level verification, and system-level reachability analysis, this case study provides a challenging and representative benchmark for future research. Advancing verification scalability, integrating physically meaningful specifications more directly into training, and developing tighter connections between controller-level and system-level guarantees remain important directions for future work.
